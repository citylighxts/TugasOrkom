\documentclass{article}
\title{Jawaban Hamacher}
\author{Hana Azizah Nurhadi}
\date{5025231134}
\begin{document}
\maketitle

\section{2.1}
Given a binary pattern in some memory location, is it possible to tell whether this pattern represents a machine instruction or a number?
\vspace{0.5cm}
\\ No; any binary pattern can be interpreted as a number or as an instruction

\section{2.2}
Consider a computer that has a byte-addressable memory organized in 32-bit words ac- cording to the big-endian scheme. A program reads ASCII characters entered at a keyboard and stores them in successive byte locations, starting at location 1000. Show the contents of the two memory words at locations 1000 and 1004 after the word “Computer” has been entered.
\vspace{0.5cm}
\\ 1000: 00000000 00000000 00000000 01000011
\\ 1004: 00000000 00000000 00000000 01101111

\section{2.4}
Registers R4 and R5 contain the decimal numbers 2000 and 3000 before each of the following addressing modes is used to access a memory operand. What is the effective address (EA) in each case?
\\ (a) 12(R4)
\\ (b) (R4,R5)
\\ (c) 28(R4,R5)
\\ (d) (R4)+
\\ (e) -(R4)
\vspace{0.5cm}
\\ (a) 2012
\\ (b) 5000
\\ (c) 8000
\\ (d) 2000
\\ (e) 1999

\section{2.5}
Write a RISC-style program that computes the expression SUM = 580 + 68400 + 80000.
\vspace{0.5cm}
\\ LOAD R1, 580
\\ LOAD R2, 68400
\\ ADD R1, R1, R2
\\ LOAD R2, 80000
\\ ADD R1, R1, R2
\\ STORE R1, SUM

\section{2.9}
Rewrite the addition loop in Figure 2.8 so that the numbers in the list are accessed in the reverse order; that is, the first number accessed is the last one in the list, and the last number accessed is at memory location NUM1. Try to achieve the most efficient way to determine loop termination. Would your loop execute faster than the loop in Figure 2.8?
\vspace{0.5cm}
\\ LOAD R1, NUM1
\\ LOAD R2, NUM1
\\ ADD R2, R2, R2

\end{document}